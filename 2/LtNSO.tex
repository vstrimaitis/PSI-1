\documentclass{VUMIFPSkursinis}
\usepackage{algorithmicx}
\usepackage{algorithm}
\usepackage{algpseudocode}
\usepackage{amsfonts}
\usepackage{amsmath}
\usepackage{booktabs}
\usepackage{blindtext}
\usepackage{bm}
\usepackage{caption}
\usepackage{color}
\usepackage{colortbl}
\usepackage{float}
\usepackage{graphicx}
\usepackage{listings}
\usepackage{multirow}
\usepackage{scrextend}
\usepackage{subfig}
\usepackage{wrapfig}
\usepackage{longtable}
\usepackage{enumitem}
\usepackage{xparse}
%\usepackage{tabularx}
\usepackage{ltxtable}
\usepackage{tabu}
\usepackage{xcolor}

% Titulinio aprašas
\university{Vilniaus universitetas}
\faculty{Matematikos ir informatikos fakultetas}
\department{Programų sistemų katedra}
\papertype{2 laboratorinis darbas}
\title{Lietuvos Nacionalinė Sporto Organizacija }
\titleineng{Lithuanian National Sports Organization}
\status{2 kurso 5 grupės studentai}
\author{Margiris Strakšys}
\secondauthor{Gabrielė Saletytė}
\thirdauthor{Vytautas Strimaitis}
\fourthauthor{Gabijus Arūnas Šukaitis}
\supervisor{lekt. Vytautas Valaitis}
\date{Vilnius – \the\year}

% Nustatymai
% \setmainfont{Palemonas}   % Pakeisti teksto šriftą į Palemonas (turi būti įdiegtas sistemoje)
%\bibliography{bibliografija}

\colorlet{headcolor}{gray!25}
\renewcommand{\tabularxcolumn}[1]{ m{#1} }
\newcommand\TopSpace{\rule{0pt}{2.6ex}}       % Top strut
\newcommand\BottomSpace{\rule[-1.2ex]{0pt}{0pt}} % Bottom strut
\newcolumntype{M}[1]{ >{\centering\arraybackslash} m{#1} }

\newcounter{subCount}
\newcounter{ir}
\setcounter{ir}{0}
\newcounter{fr}
\setcounter{fr}{0}
\newcounter{nfr}
\setcounter{nfr}{0}

\newcommand{\ir}[2]{%
    \stepcounter{ir}
    \generateTable{IR\their}{#1}{#2}
}

\newcommand{\fr}[2]{%
    \stepcounter{fr}
    \generateTable{FR\thefr}{#1}{#2}
}

\newcommand{\nfr}[2]{%
    \stepcounter{nfr}
    \generateTable{NFR\thenfr}{#1}{#2}
}

\makeatletter
\def\nobreakhline{%
  \noalign{\ifnum0=`}\fi
    \penalty\@M
    \futurelet\@let@token\LT@@nobreakhline}
\def\LT@@nobreakhline{%
  \ifx\@let@token\hline
    \global\let\@gtempa\@gobble
    \gdef\LT@sep{\penalty\@M\vskip\doublerulesep}% <-- change here
  \else
    \global\let\@gtempa\@empty
    \gdef\LT@sep{\penalty\@M\vskip-\arrayrulewidth}% <-- change here
  \fi
  \ifnum0=`{\fi}%
  \multispan\LT@cols
     \unskip\leaders\hrule\@height\arrayrulewidth\hfill\cr
  \noalign{\LT@sep}%
  \multispan\LT@cols
     \unskip\leaders\hrule\@height\arrayrulewidth\hfill\cr
  \noalign{\penalty\@M}%
  \@gtempa}
\makeatother

\newcommand{\header}[1]{%
    \nobreakhline
    \multicolumn{3}{|c|}{\cellcolor{headcolor} #1} \\*
    \nobreakhline
}

\newcommand{\generateTable}[3]{%
    #1 &
    {
        \setcounter{subCount}{1}%
        \let\oldbackslash\\
        \renewcommand{\tabularxcolumn}[1]{ p{##1} }%
        \begin{tabularx}{\linewidth}{  @{\hspace{\tabcolsep}#1.\thesubCount\ifnum\value{subCount}<10 \hphantom{0}\fi\hspace{\tabcolsep}} | X  }
          \global\def\\{\TopSpace\BottomSpace\stepcounter{subCount}\oldbackslash \hline}
          #3
        \end{tabularx}%
        \global\let\\\oldbackslash
    } &
    #2 \\
    \hline
}

%==========================================================================> Dokumento pradžia <========================================================================
\begin{document}
\maketitle

\tableofcontents

\sectionnonum{Anotacija} \label{anotacija}
	Šiuo darbu siekiama specifikuoti versle išskiriamus reikalavimus. 
	Šiame laboratoriniame bus analizuojamos išskirtų reikalavimų savybės ir jų verifikavimas.
	Projekto tikslas - išnagrinėti informacinės sistemos vartotojo sąsajos, funkcinius ir nefunkcinius reikalavimus bei jų ypatumus.
\sectionnonum{Įvadas} \label{ivadas}
  \subsection*{Programų sistemos pavadinimas} \label{ivadas_pavadinimas}
    „Lietuvos nacionalinė sporto organizcija” (sutrumpintas sistemos pavadinimas - „LtNSO”).

  \subsection*{Darbo tikslas} \label{ivadas_tikslas}
    Specifikuoti vartotojo sąsajos, funkcinius ir nefunkcinius reikalavimus.
	
  \subsection*{Naudotojai} \label{ivadas_naudotojai}
	Informacinė sistema skirta organizatoriams, komitetui, dalyviams, darbuotojams bei visiems susidomėjusiems.

  \subsection*{Darbo pagrindas}  \label{ivadas_darboPagrindas}
    Dokumentas parengtas kaip programų sistemų inžinerijos laboratorinis darbas.

% Gabriele START
\section{Vartotojo interfeiso reikalavimai} \label{vartotojoInterfeisoReikalavimai}
  Šiame skyriuje apžvelgiami ir konkretizuojami reikalavimai sistemos vartotojo interfeisui. Taip pat suformuluojamos atskirų agentų užduotys.
  \begin{longtabu} to \linewidth{ | M{2cm} | @{}X@{} | M{2.5cm} | }
    \caption{Vartotojo interfeiso reikalavimai}
    \label{table:interfeisoReikalavimai}
    \endfirsthead
    \endhead
    \hline
    \textbf{Kodas} & \multicolumn{1}{|c|}{\textbf{Reikalavimas}} & \textbf{Svarba} \\
    \hline
    \header{Dalykinės srities metaforos reikalavimai}
    \ir{Būtinas}
    {
      Sąrašas „Ieškome“ sistemoje pateikia sąrašą potencialiems darbuotojams su laisvomis darbo vietomis. \\
      Pasirinkimas „Registracija“ sistemoje pateikia anketą naujo sistemos vartotojo registracijai. \\
      Pasirinkimas „Bilietai“ pateikia likusių bilietų kiekį ir suteikia galimybę nusipirkti bilietus į norimas varžybas. \\
      Pasirinkimas „Vaizdo įrašai“ pateikia nuorodų sąrašą su įvykusių bei dabar vykstančių varžybų įrašais. \\
      Pasirinkimas „Rezultatai“ sistemoje pateikia informaciją su varžybų rezultatais. \\
      Pasirinkimas „Komentarai“ sistemoje pateikia atsiliepimų sąrašą. \\
	  Pasirinkimas „Renginių sąrašas“ sistemoje atveria langą su visomis varžybomis. \\
      Piktograma „Deganti lemputė“ atvaizduoja tekstinį langą naujai idėjai išreikšti bei pateikti komitetui. \\
      Piktograma „Kalendorius“ atvaizduoja kalendorių su artimiausiu metu vyksiančiomis varžybomis
    }

    \header{Formuluojamos vartotojo užduotys}
    \ir{Būtinas}
    {
      Užsiregistruoti į informacinę sistemą. \\
      Prisijungti prie sistemos su savo vartotojo vardu arba el. paštu ir slaptažodžiu. \\
      Peržiūrėti ir redaguoti savo profilį. \\
      Turėti prieigą prie tik vartotojams skirtų sistemos funkcijų. \\
      Turėti galimybę ištrinti savo vartotojo paskyrą iš sistemos. \\
      Pateikti siūlymą su naujomis idėjomis. \\
      Atsijungti nuo informacinės sitemos.
    }

    \header{Formuluojamos administratoriaus užduotys}
    \ir{Būtinas}
    {
      Privalo būti registruotas vartotojas. \\
      Pridėti naujas varžybas. \\
      Peržiūrėti jau esamų varžybų sąrašą ir redaguoti pasirinktų varžybų informaciją. \\
      Įrašyti į sistemą dalyvių rezultatus. \\
      Peržiūrėti ir redaguoti dalyvių rezultatų lentelę. \\
      Peržiūrėti ir redaguoti varžybų vaizdo įrašų sąrašą.
    }
        
    \header{Formuluojamos dalyvio užduotys}
    \ir{Būtinas}
    {
      Privalo būti registruotas vartotojas. \\
      Vykdyti varžybų paiešką. \\
      Sukurti komandą. \\
      Užsiregistruoti į varžybas individualiai ar su komanda. \\
      Peržiūrėti varžybų rezultatus, matyti savo rezultatą bendroje lentelėje. \\
      Peržiūrėti varžybų vaizdo įrašus.
    }

    \header{Formuluojamos žiūrovo užduotys}
    \ir{Būtinas}
    {
      Gali būti registruotas vartotojas. \\
      Įsijungti norimų žiūrėti varžybų tiesioginę transliaciją. \\
      Peržiūrėti jau vykusių varžybų vaizdo įrašus. \\
      Peržiūrėti pasibaigusių varžybų rezultatų suvestines. \\
      Palikti atsiliepimą apie varžybas. \\
      Pateikti siūlymą su naujomis idėjomis.
    }
    
    \header{Užduočių formulavimo kalbos reikalavimai}
    \ir{Būtinas}
    {
      Užduotys vykdomos internetinėje svetainėje bei mobiliojoje programėlėje. \\
      Pagrindinis informacijos perdavimo naudotojui būdas - vaizdinė medžiaga. Stengiamasi minimizuoti teksto kiekį,
		į pagalbą pasitelkti paveiksliukus bei spalvas, tačiau neprarasti aiškumo. \\
      Užduotys pasiekiamos per meniu ar šoninę aktualiausių užduočių lango dalį. \\
      Meniu patalpintas patogioje vartotojui vietoje, netrukdo turinio peržiūrai, tačiau yra visada pasiekiamas. \\
      Šoninė aktualių užduočių lango dalis matoma tik pagrindiniame puslapyje.
    }
    \ir{Būtinas}
    {
      Svetainė bei programėlė neturi būti perpildyta reklamomis. \\
      Reklamos turi būti neįkyrios. \\
      Reklamos turi netrukdyti naudotis sistema. \\
      Reklamos turi būti neinteraktyvios. \\
      Reklamos talpinamos šonuose.
    }
    \ir{Pageidautinas}
    {
      Peržiūrimas vaizdo įrašas atsidaro naujame, kiek mažesniame lange.
    }
    
    \header{Interfeiso darnos ir standartizavimo reikalavimai}
    \ir{Būtinas}
    {
      Vartotojo interfeisas turi atitikti operacinės sistemos, kurioje programa naudojama, standartą. \\
      Vartotojo interfeisas turi būti visapusiškai susietas: programos komponentai turi derėti estetiškai bei neatlikti padrikų funkcijų. 
    }

    \header{Pranešimų formulavimo reikalavimai}
    \ir{Būtinas}
    {
      Sistemoje pranešimai, skirti vartotojui, turi būti formuluojami ne ilgesni nei dvejais sakiniais. \\                                      
      Sistemos pranešimai turi apimti tik vieną temą, dėl kurios pranešimas buvo iškviestas. Kitokios informacijos juose negali būti. \\
      Pranešimai turi būti aiškūs, kad vartotojas kiekvienu atveju žinotų kaip elgtis toliau.
    }
    
    \header{Interfeiso individualizavimo reikalavimai}
    \ir{Būtinas}
    {
      Galima pasirinkti kalbą iš kelių variantų, kaip: lietuvių, anglų, rusų. \\
      Galima pasirinkti informacinės sistemos fono tematiką, spalvą, kad vartotojai galėtų pritaikyti aplinką savo akims patinkančiu variantu. \\
      Vartotojui yra galimybė pasirinkti vieną ar kelias mėgiamiausias sporto rungtis, kurias jis nori nuolatos sekti. 
		Pasirinkus jas, vos prisijungus į sistemą, pradiniame puslapyje pateikiama naujausia šių sričių informacija. \\
      Vartotas gali pasirinkti sekamus dalyvius, kad galėtų patogiai matyti su jų pasirodymu varžybose susijusią informaciją.
    }
  \end{longtabu}
%Gabriele END
%Vytas START  
\section{Funkciniai reikalavimai} \label{funkciniaiReikalavimai}
  Šiame skyriuje pateikiami sistemos funkciniai reikalavimai, t.y. pagrindinės sistemos atliekamos funkcijos, konkretūs jų aprašymai.
	\subsection*{Pagrindinės sistemos funkcijos}
  \begin{longtabu} to \linewidth{ | M{2cm} | @{}X@{} | M{2.5cm} | }
    \caption{Pagrindinės sistemos funkcijos}
    \label{table:pagrindinesSistemosFunkcijos}
    \endfirsthead
    \endhead
    \hline
    \textbf{Kodas} & \multicolumn{1}{|c|}{\textbf{Reikalavimas}} & \textbf{Svarba} \\
    \hline
	
    \header{Vartotojo registracija}
    \fr{Būtinas}
    {
      Pradiniame puslapyje paspaudus mygtuką ,,Registracija'' atsiveria registracijos anketa. \\
      Vartotojas turi galimybę tekstiniuose laukuose įvesti savo vardą, pavardę, elektroninį paštą bei vartotojo vardą, jei pageidaujama. \\
      Šalia įvesto prisijungimo vardo atsiradusi žalia varnelė simbolizuoja galimą vartotojo vardą. \\
      Šalia įvesto prisijungimo vardo atriradęs raudonas ,,X'' simbolis simbolizuoja netinkamą vartotojo vardą. Registracija nebaigiama, kol laukas nepataisomas. \\
      Vartotojas turi tekstiniame lauke įvesti slaptažodį. \\
      Vartotojui rašant slaptažodį visi simboliai privalo būti paslėpti (pvz. simboliais ,,*''). \\
      Šalia slaptažodžiui skirto lauko atsiradęs žalias užrašas ,,Stiprus'' simbolizuoja sunkiai atspėjamą slaptažodį,
			kuris yra ne trumpesnis nei 8 simbolių ilgio, jame egzituoja bent trijų skirtingų simbolių tipai, tokie kaip
			raidės, skaičiai, specialieji simboliai. \\
		  Šalia slaptažodžiui skirto lauko atsiradęs raudonas užrašas ,,Silpnas'' simbolizuoja netinkamą slaptažodį,
			kuris nėra bent 8 simbolių ilgio arba kuriame yra tik vieno tipo simboliai. Registracija nebaigiama, kol slaptažodis nepakeičiamas į stiprų. \\
      Po slaptažodžiui skirto lauko turi būti kitas laukas, skirtas pakartoti slaptažodžiui. \\
      Šalia antrojo slaptažodžiui skirto lauko atsiradusi žalia varnelė simbolizuoja sutapusį slaptažodį. \\
      Šalia antrojo slaptažodžiui skirto lauko atsiradęs raudonas ,,X'' simbolizuoja nesutapusį slaptažodį. \\
    }
	
    \header{Komandos sukūrimas}
    \fr{Būtinas}
    {
      Vartotojui prisijungus į sistemą ir pradinio puslapio viršutiniame dešiniame kampe paspaudus mygtuką su profilio ikona atsiveria
      vartotojo profilio duomenys. \\
      Po esančiais duomenimis egzistuoja skiltis ,,Mano komandos''.\\
      Po skilties su vartotojo komandų sąrašu yra mygtukas ,,Sukurti komandą''. \\
      Paspaudus mygtuką ,,Sukurti komandą'' atsiveria langas komandos registracijai. \\
      Komandos registracijos lango viršuje yra tekstinis laukas, skirtas komandos pavadinimui. \\
      Po pavadinimo esančiame tekstiniame lauke įrašomas komandos aprašymas. \\
      Po aprašymo esančiuose tekstiniuose laukuose įrašomi kitų komandos narių vartotojo vardas (vardai) arba elektroninio pašto adresas (adresai). \\
      Norint pridėti papildomą komandos narį, spaudžiamas mygtukas ,,+'', esantis po paskutinio komandos nario informacija. \\
      Įvedus visus narius spaudžiamas mygtukas ,,Sukurti''. \\
      Sukūrus komandą, kiti nariai turi patvirtinti prisijungimą. \\
      Kol narys nėra patvirtinęs prisijungimo prie komandos, jo vardas atvaizduojamas šviesiai pilka spalva, šalia vaizduojama būsena ,,Laukia patvirtinimo''. \\
      Kai narys patvirtina prisijungimą, jo vardas tampa atvaizduojamas juoda spalva, šalia atsiranda būsena ,,Narys''. \\
      Jei narys atmeta pakvietimą, jo vardas dingsta iš komandos sąrašo. \\
      Šalia komandos kūrėjo vardo vaizduojama būsena ,,Kapitonas''.	
    }
	
    \header{Prisijungimas prie komandos}
    \fr{Būtinas}
    {
      Vartotojui prisijungus į sistemą ir pradinio puslapio viršutiniame dešiniame kampe paspaudus mygtuką su profilio ikona atsiveria
      vartotojo profilio duomenys. \\
      Esančių duomenų pabaigoje vaizduojamos vartotojo komandos bei komandos, į kurias vartotojas yra kviečiamas. \\
      Paspaudus ant komandos, į kurią yra kviečiamas, vartotojas gali peržiūrėti narių sudėtį bei prisijungti prie komandos arba atmesti pakvietimą
      paspaudus atitinkamai žalią mygtuką ,,Prisijungti'' arba raudoną mygtuką ,,Atmesti''. \\
      Prisijungus prie komandos, ji atsiduria sąraše ,,Mano komandos''.
    }
    
    \header{Dalyvių registracija į varžybas}
    \fr{Būtinas}
    {
      Pradiniame puslapyje pasirinkus ,,Renginių sąrašas'' atsiveria puslapis su visais artėjančiais sporto renginiais. \\
      Pasirinkus norimą renginį, atsiveria galimybė registruotis į varžybas, jei vartotojas yra prisijungęs. \\
      Jei vartotojas nėra prisijungęs, jo paprašoma prisijungti prieš registruojantis į renginį. \\
      Jei registracija į varžybas jau yra pasibaigusi ar nebėra laisvų vietų, mygtukas ,,Registracija'' vaizduojamas pilkai, nėra galimybės jo paspausti. \\
      Jei registracija dar vyksta ir varžybos skirtos individualiems dalyviams, paspaudus raudoną mygtuką ,,Registracija'' dalyvis užregistruojamas į renginį. \\
      Jei registracija dar vyksta ir varžybos yra komandinės, tai paspaudus ,,Registracija'' parodomas komandų, su tinkama sudėtimi, sąrašas. \\
      Pasirinkus norimą komandą varžyboms, komanda užregistruojama į sporto renginį. \\
      Skiltyje ,,Mano profilis'' pateikiamas visų renginių, į kuriuos yra užsiregsitravęs vartotojas, sąrašas.
    }
    
    \header{Darbuotojų registracija}
    \fr{Būtinas}
    {
      Pradinio puslapio dešinėje yra skiltis ,,Ieškome'', kurioje pateiktas ieškomų darbuotojų sąrašas. \\
      Šalia kiekvieno punkto yra mygtukas ,,Registracija'', kurį paspaudus atsiveria galimybė įkelti CV bei motyvacinį laišką. \\
      Skiltyje ,,Mano profilis'' pateikiamas sąrašas pozicijų, į kurias buvo aplikuota, ir kiekvienos iš jų būsena.
    }
    
    \header{Bilietų prekyba}
    \fr{Būtinas}
    {
      Pradiniame puslapyje kairėje esantis pasirinkimas ,,Bilietai'' atveria puslapį, kuriame pateikiama informacija apie likusius bilietus. \\
      Atsivėrusiame puslapyje galima pasirinkti norimas varžybas, bilietų kiekį bei vietas. \\
      Puslapio apačioje vaizduojamos įvairių bankų ikonos, kurios žiūrovui leidžia pasirinkti patogiausią apmokėjimo būdą.
    }
    
    \header{Tiesioginis renginio stebėjimas internetu}
    \fr{Būtinas}
    {
      Pradiniame puslapyje kairėje esantis mygtukas ,,Vaizdo įrašai'' atveria nuorodų sąrašą su įvykusių varžybų nuorodomis į vaizdo įrašus. \\
      Virš sąrašo yra raudonas mygtukas su užrašu “Gyvai”. \\
      Paspaudus mygtuką ,,Gyvai'', vartotojui pateikiamas sąrašas su dabar vykstančių rungčių transliacijomis. \\
      Spustelėjęs ant norimų varžybų, vartotojas gali stebėti renginį realiu laiku. \\
    }
    
    \header{Rezultatų peržiūra}
    \fr{Būtinas}
    {
      Sistemos vartotojui, pradiniame puslapyje pasirinkusiam “Rezultatai” lentelės forma pateikiamos visos varžybos su rezultatais. \\
      Jei varžybos dar neįvyko arba rezultatai dar nėra įkelti į sistemą, šalia varžybų vietoje rezultatų privalo būti pateiktas pilkas užrašas: 
      ,,Rezultatų nėra''. \\
    }
    
    \header{Komentarų skaitymas ir rašymas}
    \fr{Būtinas}
    {
      Vartotojui pradiniame puslapyje paspaudus mygtuką “Komentarai” jam privalo būti atvaizduotas visų jau parašytų atsiliepimų sąrašas. \\
      Atsiliepimų sąrašo pabaigoje turi egzistuoti tekstinis laukas su galimybe palikti savo atsiliepimą. \\
      Virš šio lauko turi būti naudotojui prieinamos galimybės tą tekstą redaguoti.
    }
%Vytas END
%Margiris START
    
    \header{Naujos idėjos pasiūlymas}
    \fr{Būtinas}
    {
      Pradiniame puslapyje vartotojui spustelėjus ikoną su degančia lempute atsiveria tekstinis laukas, kuriame žiūrovas gali detaliai aprašyti savo idėją,
      kurią norėtų pateikti organizatoriams. \\
      Virš šio tekstinio lauko turi būti vieta temai bei naudotojo el. paštui, jei jis nėra prisijungęs prie sistemos. \\
      Ši informacija automatiškai pateikiama organizatoriams.
    }
    
    \header{Varžybų grafiko peržiūra}
    \fr{Būtinas}
    {
      Pradinio puslapio dešinėje turi būti atvaizduotas kalendorius su visomis varžybomis. \\
      Paspaudus ant šio kalendoriaus, jis atsiveria per visą langą su detalesne informacija. \\
      Spustelėjus ant kurios nors konkrečios dienos, iššoka langas su informacija apie atitinkamas rungtis: 
      jų laikas, vieta, dalyviai, parduotų bilietų kiekis procentais, būsena (“pasibaigęs”, “vyksta”, “artėja”) ir kt.
    }
	\end{longtabu}
\newpage
\subsection*{Pagalbinės sistemos funkcijos}
  \begin{longtabu} to \linewidth{ | M{2cm} | @{}X@{} | M{2.5cm} | }
    \caption{Pagalbinės sistemos funkcijos}
    \label{table:pagalbinesSistemosFunkcijos}
    \endfirsthead
    \endhead
    \hline
    \textbf{Kodas} & \multicolumn{1}{|c|}{\textbf{Reikalavimas}} & \textbf{Svarba} \\
    \hline
    
    \header{Gražiausių savaitės momentų peržiūra}
    \fr{Pageidautinas}
    {
      Pradinio puslapio kairėje yra mygtukas ,,Vaizdo įrašai'', kuris atveria langą su varžybų įrašais. \\
      Lango pabaigoje yra mygtukas ,,Savaitės gražiausi epizodai'', atveriantis šios savaitės gražiausių epizodų rinkinį. \\
      Rinkinio pabaigoje matomi praėjusių savaičių rinkiniai su galimybe juos peržiūrėti.
    }

    \header{Savaitės geriausio sportininko rinkimai}
    \fr{Pageidautinas}
    {
      Pagrindiniame lange paspaudus piktogramą su auksiniu užrašu „MVP” (most valuable player) ir užrašu „PGŽ” (pats geriausias žaidėjas) 
      atsidarys naujas langas, kuriame vyks balsavimas. \\
      Balsuoti galima už bet kurį žaidėją, dalyvaujantį varžybose šią savaitę. \\
      Galima balsuoti tris kartus, taip suteikiant galimybę vartotojui diferencijuoti savo pasirinkimą, jeigu jis turi ne vieną favoritą. \\
      Balsavimas vyksta rodant žaidėjų sąrašą, o įvedandus tekstą pasirinkimų mažės, kol galiausiai vartojas pasirinks savo favoritą. \\
      Tuomet atsiras pasirinkimo mygtukas „Patvirtinti pasirinkimą”. \\
      Pasirinkus žaidėją tris kartus, atsiras lentelė su sakiniu „Jau balsavote 3 kartus, bandykite kitą savaitę”.
    }
    
    \header{Turnyrinių lentelių archyvas}
    \fr{Pageidautinas}
    {
      Pagrindiniame lange paspaudus piktogramą su senų knygų bei rankraščių atvaizdais atsivers naujas langas, kuriame galima pasirinkti norimą sporto šaką. \\
      Pasirinkus šaką atsivers dar vienas naujas langas, kuriame bus sąrašas su visais įvykusiais tos sporto šakos turnyrais. \\
      Turi būti galima filtravimo funkcija, kur vartotojas gali įvesti datą arba turnyro pavadinimą. \\
      Paspaudus ant turnyro atsiveria atitinkamo turnyro lentelė. \\
      Lentelėje matysis visi vykusio turnyro rezultatai, dalyvavę žaidėjai ir teisėjų brigada. 
    }
    
    \header{Turnyrinių lentelių generacija}
    \fr{Pageidautinas}
    {
      Dalyviams užsiregistravus ir administratoriui patikrinus visus žaidėjus automatiškai sudaroma turnyrinė lentelė. \\
      Galimi variantai: šveicariškas turnyras, vieno „minuso” arba pralaimėjimo turnyras, dviejų „minusų” arba pralaimėjimų turnyras, 
      grupių turnyras ir rato sistemos turnyras. \\
      Sistema, priklausomai nuo to, kiek yra dalyvaujančių komandų, nustato, kokio formato imti turnyro dydį.
    }
	\end{longtabu}
	
\section{Nefunkciniai reikalavimai} \label{nefunkciniaiReikalavimai}
  Šiame skyriuje pateikiami nefunkciniai reikalavimai sistemoms, t.y. reikalavimai, tiesiogiai nesusiję su sistemos atliekamomis funkcijomis.
	\subsection*{Vidinių interfeisų reikalavimai}
    \begin{longtabu} to \linewidth{ | M{2cm} | @{}X@{} | M{2.5cm} | }
      \caption{Vidinių interfeisų reikalavimai}
      \label{table:VidiniuInterfeisuReikalavimai}
      \endfirsthead
      \endhead
      \hline
      \textbf{Kodas} & \multicolumn{1}{|c|}{\textbf{Reikalavimas}} & \textbf{Svarba} \\
      \hline
      \header{Operacinės sistemos naudojimo reikalavimai}
      \nfr{Būtinas}
      {
        Įrenginys turi turėti prieigą prie interneto. \\
        Kompiuteris turi turėti interneto naršyklę, o mobilusis įrenginys gali turėti ir LtNSO mobiliąją programėlę. 
      }
    
      \header{Sąveikos su duomenų bazėmis reikalavimai}
      \nfr{Būtinas}
      {
        Naudojantis MSSQL duomenų valdymo sistema duomenys yra išsaugomi reliaciniu būdu. \\
        Duomenų bazėje yra saugomos turnyrinės lentelės, turnyrų dalyviai, vartotojų duomenys, vaizdo įrašai, informacija apie būsimus turnyrus,
        informacija apie bilietų pardavimą, komentarai bei pateikti pasiūlymai.
      }
      
      \header{Dokumentų mainų reikalavimai}
      \nfr{Būtinas}
      {
        Informacinė sistema bendrauja su serveriu JSON formatu.
      }
      
      \header{Darbo kompiuterių tinkluose reikalavimai}
      \nfr{Būtinas}
      {
        Informacinė sistema keičiasi duomenimis per HTTP protokolą.
      }
      
      \header{Programavimo aplinkos reikalavimai}
      \nfr{Būtinas}
      {
        Darbui su serverine programine įranga naudojama Microsoft Visual Studio IDE. \\
        Internetinė svetainė kuriama Adobe Dreamweaver. \\
        Darbui su duomenų baze naudojama Microsoft SQL Server Management Studio. \\
        Programėlė, skirta Android OS, kuriama su Android Studio. \\
        Programėlė, skirta iOS, kuriama su XCode. \\
        Kodo versijavimui ir dalinimuisi naudojama GitHub repozitorija. \\
        Informacinės sistemos serverio pusė programuojama C\# kalba. \\
        Internetinei svetainei sukurti naudojamos naujausios web technologijos.
      }
    \end{longtabu}
%Margiris END
%Gabijus START
\subsection*{Veikimo reikalavimai}
  \begin{longtabu} to \linewidth{ | M{2cm} | @{}X@{} | M{2.5cm} | }
    \caption{Veikimo reikalavimai}
    \label{table:veikimoReikalavimai}
    \endfirsthead
    \endhead
    \hline
    \textbf{Kodas} & \multicolumn{1}{|c|}{\textbf{Reikalavimas}} & \textbf{Svarba} \\
    \hline
	
    \header{Vaizdavimo tikslumo reikalavimai}
    \nfr{Būtinas}
    {
      Data informacinėje sistemoje atvaizduojama YYYY-MM-DD formatu. \\
      Programų sistemos palaikymui yra būtina bent 360 x 640 ekrano rezoliucija su 16M spalvų. \\
    }
    
    \header{Skaičiavimų tikslumo reikalavimai}
    \nfr{Būtinas}
    {
      Skaičiavimo tikslumus rungtyse nustato rungties teisėjai arba laikomasi tos sporto šakos federacijos nustatytomis tikslumo taisyklėmis. \\
      Skaičiavimai visuose rezultatuose turi būti pastovūs.
    }
    
    \header{Patikimumo reikalavimai}
    \nfr{Būtinas}
    {
      Informacinės sistemos patikimumas yra nurodomas atsižvelgiant į sistemos veikimo be sutrikimų laiką. \\
      Informacinėje sistemoje gali pasitaikyti ne daugiau kaip vienas sutrikimas per vienerius metus. \\
      Sutrikimas turi būti pašalintas per vieną valandą nuo jo atsiradimo pradžios.
    }
    
    \header{Robastiškumo reikalavimai}
    \nfr{Būtinas}
    {
      Parodyti pranešimą, kai interneto ryšys tampa silpnas arba iš vis dingsta. \\
      Būtina nuolatos daryti atsargines duomenų kopijas. \\
      Įvykus sistemos sutrikimui vartotojo darbo funkcionalumas turi būti atkurtas greičiau nei per 20 sekundžių.
    }
    
    \header{Našumo reikalavimai}
    \nfr{Būtinas}
    {
      Didžiausia leistina programų sistemos apkrova yra 10 000 vartotojų, prisijungusių vienu metu. \\
      Reakcijos laikas į sistemos vartotojo, dalyvio atliekamus veiksmus turi būti ne daugiau kaip 1 sekundė. \\
      Užklausos vykdymo laikas turi būti ne daugiau nei viena sekundė.
    }
	\end{longtabu}
	
\subsection*{Diegimo reikalavimai}
  \begin{longtabu} to \linewidth{ | M{2cm} | @{}X@{} | M{2.5cm} | }
    \caption{Diegimo reikalavimai}
    \label{table:diegimoReikalavimai}
    \endfirsthead
    \endhead
    \hline
    \textbf{Kodas} & \multicolumn{1}{|c|}{\textbf{Reikalavimas}} & \textbf{Svarba} \\
    \hline
	
    \header{Ruošinio reikalavimai}
    \nfr{Būtinas}
    {
      Informacinė sistema pateikiama Google Play ir iTunes aplikacijų parduotuvėse, iš kurios vartotojai, 
      administratorius ir dalyviai aplikaciją gali atsisiųsti nemokamai. \\
      Įsirašius programėlę ir ją pirmą kartą paleidus automatiškai atsidarys registracijos langas, kurį, jei vartotojas nenori užsiregistruoti, 
      galima uždaryti.
    }
    
    \header{Instaliavimo reikalavimai}
    \nfr{Būtinas}
    {
      Programa turi būti instaliuojama automatiškai per įrenginį. \\
      Įrenginys turi turėti pakankamai atminties programėlės įrašymui.
    }
    
    \header{Pradinio duomenų bazių kaupimo reikalavimai}
    \nfr{Būtinas}
    {
      Pradinėje duomenų bazėje turi būti sukurtos lentelės, į kurias bus įrašinėjami nauji duomenys. \\
      Pradinėje duomenų bazėje turi būti įrašytas bent vienas būsimas turnyras.
    }

    \header{Sistemos įsisavinamumo reikalavimai}
    \nfr{Būtinas}
    {
      Sistema turi turėti tris kalbas: lietuvių, anglų ir rusų. \\
      Ikonos savo prasme tiesiogiai atspindi mygtuko esmę sistemoje. \\
      Jei vartotojui vis dėlto kyla neaiškumų, susijusių su programos veikimu, jis gali kreiptis į konsultantą telefonu arba el.paštu.
    }

	\end{longtabu}
  
\subsection*{Aptarnavimo ir priežiūros reikalavimai}
  \begin{longtabu} to \linewidth{ | M{2cm} | @{}X@{} | M{2.5cm} | }
    \caption{Aptarnavimo ir priežiūros reikalavimai}
    \label{table:aptarnavimoIrPrieziurosReikalavimai}
    \endfirsthead
    \endhead
    \hline
    \textbf{Kodas} & \multicolumn{1}{|c|}{\textbf{Reikalavimas}} & \textbf{Svarba} \\
    \hline

	\nfr{Būtinas}
	{
		Į vartotojo užduodamus klausimus arba užklausas darbo metu turi būti atsakyta ne vėliau nei per 30 min. , o ne darbo metu ne vėliau nei per 12 valandų. \\
		Praplėtus informacinės sistemos funkcionalumą, kad būtų užtikrintas patikimumas, būtina testuoti atnaujinimus prieš leidžiant jais naudotis vartotojams, dalyviams ir administratoriui. \\
		Atlikus testavimą ir nesusidūrus su jokiomis klaidomis vartotojai, dalyviai  ir administratorius turi būti automatiškai informuoti pranešimu apie pasikeitimus.\\
		Programėlę galima atsinaujinti per GooglePlay arba iTunes aplikacijų parduotuves.
	}
	
  \end{longtabu}
	
\subsection*{Tiražuojamumo reikalavimai}
  \begin{longtabu} to \linewidth{ | M{2cm} | @{}X@{} | M{2.5cm} | }
    \caption{Tiražuojamumo reikalavimai}
    \label{table:tirazuojamumoReikalavimai}
    \endfirsthead
    \endhead
    \hline
    \textbf{Kodas} & \multicolumn{1}{|c|}{\textbf{Reikalavimas}} & \textbf{Svarba} \\
    \hline
	
	\nfr{Būtinas}
	{
		Programėlė turi sėkmingai veikti mobiliuosiuose įrenginiuose su Android OS, iOS. \\
		Informacinė sistema privalo palaikyti funkcionalumą šiose naršyklėse: Google Chrome, Mozilla Firefox, Microsoft Explorer, Microsoft Edge, Safari, Opera.
	}
	
  \end{longtabu}

\subsection*{Apsaugos reikalavimai}
  \begin{longtabu} to \linewidth{ | M{2cm} | @{}X@{} | M{2.5cm} | }
    \caption{Apsaugos reikalavimai}
    \label{table:apsaugosReikalavimai}
    \endfirsthead
    \endhead
    \hline
    \textbf{Kodas} & \multicolumn{1}{|c|}{\textbf{Reikalavimas}} & \textbf{Svarba} \\
    \hline
	
    \nfr{Būtinas}
    {
      Duomenų bazė visada turi savo kopiją. Ji atnaujinama kiekvieną dieną 04:20. \\
      Įvedant slaptažodį, jo raidės atvaizduojamos juodais taškais. \\
      Duomenų bazėje slaptažodžiai saugomi iš pradžių juos užkodavus. \\
      Prie duomenų bazėje esančių konfidencialių asmens duomenų gali prieiti tik autorizuoti asmenys. 
    }
	
  \end{longtabu}  

\subsection*{Juridiniai reikalavimai}
  \begin{longtabu} to \linewidth{ | M{2cm} | @{}X@{} | M{2.5cm} | }
    \caption{Juridiniai reikalavimai}
    \label{table:juridiniaiReikalavimai}
    \endfirsthead
    \endhead
    \hline
    \textbf{Kodas} & \multicolumn{1}{|c|}{\textbf{Reikalavimas}} & \textbf{Svarba} \\
    \hline
	  \nfr{Būtinas}
    {
      Informacinė sistema turi atitikti visus Lietuvos Respublikoje galiojančius įstatymus.\\
      Informacinės sistemos duomenų bazėje esantys vartotojų registracijos ir asmeniniai duomenys yra visapusiškai apsaugoti ir prieinami tik autorizuotiems asmenims.\\
      Vartotojas užsiregistruodamas privalo sutikti su visomis informacinės sistemos naudojimosi nuostatomis.
    }
  \end{longtabu}  
%Gabijus END
\sectionnonum{Literatūros sąrašas} \label{literaturosSarasas}
	\begin{itemize}
    \item Doc. dr. K. Petrausko Programų Sistemų Inžinerijos kurso konspektai
    \item A. Abran, J. W. Moore, P.Bourque, R. Dupuis, L. L. Tripp - ,,Guide to the Software Engineering Body of Knowledge''
    \item V. Strimaitis, G. Saletytė, M. Strakšys, G. A. Šukaitis - ,,Lietuvos Nacionalinė Sporto Organizacija. Pirmasis laboratorinis darbas''. VU, MIF, Vilnius. 2016
	  \item Latex dokumentacija: \url{http://www.latex-project.org/help/documentation/}
  \end{itemize}
\end{document}
